\documentclass[12pt]{article}

\usepackage[a4paper, total={6in, 8in}]{geometry}
\usepackage{setspace,amsmath,pgfplots,subcaption,graphicx,color,subfig,tikz,tikzscale,filecontents,tcolorbox,soul,algpseudocode,algorithm,float}

\renewcommand{\thesection}{\Roman{section}} 
\newtheorem{definition}{Definition}[section]
\newtheorem{lemma}{Lemma}[section]



\begin{document}

\pagenumbering{arabic}
\doublespacing


\section{Derive Dupire Local Vol PDE}
1. Dupire local volatility SDE is defined as: \par
\begin{equation}\label{eq:LocVolSDE}
    \mathrm{d}S_t = (r_d(t)-r_f(t))S_t\mathrm{d}t + \sigma(t,S_t)\mathrm{d}W_t
\end{equation}
\begin{definition}[Fokker-Planck Equation]
For a SDE \par
\begin{equation}
    \mathrm{d}X_t = \mu(X_t,t)\mathrm{d}t + \sigma(X_t,t)\mathrm{d}W_t
\end{equation}
with drift $\mu(X_t,t)$ and diffusion coefficient $D(X_t,t)=\sigma^2(X_t,t)/2$, the Fokker-Planck equation for the probability density $p(x,t)$ of the random variable $X_t$ is \par
\begin{equation}
    \frac{\partial}{\partial t}p(x,t) = -\frac{\partial}{\partial x}[\mu(x,t)p(x,t)]+\frac{\partial^2}{\partial x^2}[D(x,t)p(x,t)].
\end{equation}
\end{definition}

2. So we get the Fokker-Planck equation of spot price $S_t$ process (a spot process following local volatility SDE diffusion) as following (we define $\mu(t)=r_d(t)-r_f(t)$):\par
\begin{equation}
    \frac{\partial}{\partial t}p(S_t,t) = -\frac{\partial}{\partial x}\left[\mu(t)\cdot S_t\cdot p(S_t,t)\right]+\frac{\partial^2}{\partial x^2}\left[\frac{1}{2}\sigma^2(S_t,t)\cdot S^2_t\cdot p(x,t)\right].
\end{equation}

3. From option price definition derive its derivatives. \par
Assume $C(T,K)$ is the Call price at time $t$, and define $DF(r_d(T))=\int_t^T e^{-r_d(u)}\mathrm{d}u$.
\begin{equation}
    C(K,T)=DF(r_d(T)) \int_0^\infty Max(S_T-K)p(S_T,T)\mathrm{d}S_T,
\end{equation}
\begin{equation}
    \frac{\partial C(K,T)}{\partial K}=-DF(r_d(T)) \int_K^\infty p(S_T,T)\mathrm{d}S_T,
\end{equation}
\begin{equation}
    \frac{\partial^2 C(K,T)}{\partial K^2}=DF(r_d(T)) \cdot p(K,T),
\end{equation}
\begin{equation}
\begin{split}
    \frac{\partial C(K,T)}{\partial T}=&-r_d(T)C(K,T)+DF(r_d(T)) \int_0^\infty Max(S_T-K) \frac{\partial p(S_T,T)}{\partial T}\mathrm{d}S_T \\
    =&-r_d(T)C(K,T)+DF(r_d(T)) \int_K^\infty (S_T-K) \\
    &\left\{-\frac{\partial[\mu(T) S_T\cdot p(S_T,T)]}{\partial S_T}+\frac{\partial^2 [0.5\sigma^2(S_T,T)\cdot S^2_T\cdot p(S_T,T)]}{\partial S^2_T}\right\}\mathrm{d}S_T \\
\end{split}
\end{equation}
4. Two useful identities. \par
\begin{equation}
    I_1 = -\mu(T) \int_K^\infty (S_T-K)\left\{\frac{\partial[S_T\cdot p(S_T,T)]}{\partial S_T}\right\}\mathrm{d}S_T
\end{equation}
\begin{equation}
    I_2 = \int_K^\infty (S_T-K)\left\{\frac{\partial^2 [0.5\sigma^2(S_T,T)\cdot S^2_T\cdot p(S_T,T)]}{\partial S^2_T}\right\}\mathrm{d}S_T
\end{equation}
For $I_1$, because \par
\begin{equation}
\begin{split}
    \frac{C(K,T)}{DF(r_d(T))} &= \int_K^\infty (S_T-K)p(S_T,T)\mathrm{d}S_T \\
    &= \int_K^\infty S_Tp(S_T,T)\mathrm{d}S_T - K\int_K^\infty p(S_T,T)\mathrm{d}S_T \\
    &= \int_K^\infty S_Tp(S_T,T)\mathrm{d}S_T + \frac{K}{DF(r_d(T))}\frac{\partial C(K,T)}{\partial K}
\end{split}
\end{equation}
\begin{equation}
    \int_K^\infty S_Tp(S_T,T)\mathrm{d}S_T = \frac{C(K,T)}{DF(r_d(T))} - \frac{K}{DF(r_d(T))}\frac{\partial C(K,T)}{\partial K}
\end{equation}
so define $u = S_T - K$, $u'= 1$, $v'=\frac{\partial}{\partial S_T}[S \cdot p(S_T,T)]$, $v=S \cdot p(S_T,T)$, \par
\begin{equation}
\begin{split}
    I_1 &= \mu(T)(S_T-K)S_T p(S_T,T)\Big|^{\infty}_{K}-\mu(T)\int_K^\infty S_Tp(S_T,T)\mathrm{d}S_T \\
        &=0-0-\mu(T)\int_K^\infty S_Tp(S_T,T)\mathrm{d}S_T \\
        &=-\frac{\mu(T)C(K,T)}{DF(r_d(T))} + \frac{\mu(T)K}{DF(r_d(T))}\frac{\partial C(K,T)}{\partial K}.
\end{split}
\end{equation}
For $I_2$, because
\begin{equation}
    p(K,T)=\frac{1}{DF(r_d(T))}\frac{\partial^2 C}{\partial K^2},
\end{equation}
so use integral by parts with $u=S_T-K$, $u'=1$, $v'=\frac{\partial^2}{\partial S^2}[\sigma^2(S_T,T)\cdot S^2_T\cdot p(S_T,T)]$, $v=\frac{\partial}{\partial S}[\sigma^2(S_T,T)\cdot S^2_T\cdot p(S_T,T)]$,
\begin{equation}
\begin{split}
    I_2 &= 0.5(S_T-K)\frac{\partial}{\partial S}\left[\sigma^2(S_T,T) S^2_T p(S_T,T)\right]\Big|^{\infty}_{K} - 0.5\int_K^\infty \frac{\partial}{\partial S}\left[\sigma^2(S_T,T) S^2_T p(S_T,T)\right] \mathrm{d}S_T \\
        &= 0 - 0 - 0.5\sigma^2(S_T,T) S^2_T p(S_T,T)\Big|^\infty_K\\
        &= \frac{0.5\sigma^2(K,T)K^2}{DF(r_d(T))}\frac{\partial^2 C}{\partial K^2}.
\end{split}
\end{equation}
5. Use the two integrals, we can get 
\begin{equation}
\begin{split}
    \frac{\partial C(K,T)}{\partial T}= -r_f(T)C(K,T) - [r_d(T)-r_f(T)]K\frac{\partial C(K,T)}{\partial K} + \frac{1}{2}\sigma^2(K,T)K^2\frac{\partial^2 C}{\partial K^2}.
\end{split}
\end{equation}

\section{Derive Drift-less Dupire Local Vol PDE}
Initially, we have $C(K,T)$. \par
Then we define normalized strike, $k=\ln(\frac{K}{F})$, or $K=F\cdot e^k$.
\begin{equation}
    \frac{\partial K}{\partial k}=K
\end{equation}
For the change of variable you need to write first:
\begin{equation}
    \tilde{C}(k,T)=C(K,T)
\end{equation}
Differentiating with respect to $k$,
\begin{equation}
    \frac{\partial \tilde{C}(k,T)}{\partial k}=\frac{\partial C}{\partial K}\frac{\partial K}{\partial k}=K\frac{\partial C}{\partial K},
\end{equation}
\begin{equation}
\begin{split}
    \frac{\partial \tilde{C}(k,T)}{\partial k^2}&=\frac{\partial}{\partial k}(K)\cdot\frac{\partial C}{\partial K}+K\cdot\frac{\partial}{\partial k}(\frac{\partial C}{\partial K}) \\
    &=K\cdot\frac{\partial C}{\partial K}+K^2\cdot\frac{\partial^2 C}{\partial K^2}.
\end{split}
\end{equation}
Differentiating with respect to T, because $K=Fe^k$,
\begin{equation}
    \frac{\partial \tilde{C}(k,T)}{\partial T}=\frac{\partial C}{\partial K}\frac{\partial K}{\partial T}+\frac{\partial C}{\partial T}
\end{equation}
From Dupire local vol PDE equation, $\frac{\partial C}{\partial T}$ is
\begin{equation}
    \frac{\partial C}{\partial T}(K,T)=-r_fC(K,T)-(r_d(T)-r_f(T))K\frac{\partial C}{\partial K}(K,T)+\frac{\sigma^2(K,T)}{2}K^2\frac{\partial^2C}{\partial K^2}(K,T)
\end{equation}
Substitute $\frac{\partial C}{\partial T}$ back to the original $\frac{\partial \tilde{C}}{\partial T}$ equation, we get 
\begin{equation}
    \frac{\partial \tilde{C}}{\partial T}(k,T)=-r_fC(K,T)+\frac{\sigma^2(K,T)}{2}K^2\frac{\partial^2C}{\partial K^2}(K,T)
\end{equation}
and finally give the `driftless' PDE by substituting back $\frac{\partial^2C}{\partial K^2}$ in term of $\frac{\partial \tilde{C}}{\partial k}$ and $\frac{\partial^2 \tilde{C}}{\partial k^2}$
\begin{equation}\label{eq:PDE_tildeC}
    \frac{\partial \tilde{C}}{\partial T}(k,T)=-r_f(T)\tilde{C}(k,T)+\frac{\sigma^2(k,T)}{2}\left[\frac{\partial^2 \tilde{C}}{\partial k^2}(k,T)-\frac{\partial \tilde{C}}{\partial k}(k,T)\right].
\end{equation}
To get rid of the `reaction' term $-r_f\tilde{C}(k,T)$ you need to convert the prices into `scaled' ones:
\begin{equation}
    \tilde{C}(k,T)=\lambda_d(0,T)F\cdot c(k,T),
\end{equation}
\begin{equation}
    c(k,T)=N(d_1)-\frac{K}{F}N(d_2)=N(d_1)-e^kN(d_2),
\end{equation}
where $\lambda_d(0,T)=\int_0^T e^{-r_d(u)} \mathrm{d}u$.
The derivatives of $\tilde{C}$ can convert to:
\begin{equation}
    \frac{\partial \tilde{C}}{\partial k}=\lambda_d(0,T)F\frac{\partial c}{\partial k},
\end{equation}
\begin{equation}
    \frac{\partial^2 \tilde{C}}{\partial k^2}=\lambda_d(0,T)F\frac{\partial^2 c}{\partial k^2},
\end{equation}
\begin{equation}
    \frac{\partial \tilde{C}}{\partial T}=-r_f(T)\lambda_d(0,T)F\cdot c(k,T)+\lambda_d(0,T)F\frac{\partial c}{\partial T}.
\end{equation}
If we inject it into equation \ref{eq:PDE_tildeC} and cancel $\lambda_d(0,T)F$, we can get:
\begin{equation}
\begin{split}
    -r_f(T)\lambda_d(0,T)F\cdot c+&\lambda_d(0,T)F\frac{\partial c}{\partial T} \\
    &=-r_f(T)\lambda_d(0,T)F\cdot c+\frac{\sigma^2(k,T)}{2}\left[\lambda_d(0,T)F\frac{\partial^2 c}{\partial k^2}-\lambda_d(0,T)F\frac{\partial c}{\partial k}\right]
\end{split}
\end{equation}
\begin{equation}
    \Rightarrow \quad \frac{\partial c}{\partial T}=\frac{\sigma^2(k,T)}{2}\left(\frac{\partial^2 c}{\partial k^2}-\frac{\partial c}{\partial k}\right)
\end{equation}
where $c(k,T)=N(d_1)-e^kN(d_2)$.



\section{Local Vol Calibration Process}
1. Drift-less Dupire PDE equation:
\begin{equation}
    \frac{\partial c}{\partial T}=\frac{\sigma^2(k,T)}{2}\left(\frac{\partial^2 c}{\partial k^2}-\frac{\partial c}{\partial k}\right)
\end{equation}
Putting $\sigma(k,T) = \sigma^{BS}(k^j_i,T_i)$ or $\sigma^{LV}(k^j_i,T_i)$, we can calibrate the forward PDE for numerical Black Scholes and Local volatilities.\par
2. Discretize on Strike Axis.
\begin{equation}
    \frac{\partial c(k^j_i,T_i)}{\partial k}=\frac{c(k^{j+1}_i,T_i)-c(k^{j-1}_i,T_i)}{k^{j+1}_i-k^{j-1}_i}+O(h)=\frac{c(k^{j+1}_i,T_i)-c(k^{j-1}_i,T_i)}{2h}+O(h^2)
\end{equation}
\begin{equation}
    \frac{\partial^2 c(k^j_i,T_i)}{\partial k^2}=\frac{c(k^{j+1}_i,T_i)-2c(k^{j}_i,T_i)+c(k^{j-1}_i,T_i)}{h^2}+O(h^2)
\end{equation}

\subsection{Explicit Scheme}
1. Discretize on Maturity Axis.
\begin{equation}
    \frac{\partial c(k^j_i,T_i)}{\partial T}=\frac{c(k^j_{i+1},T_{i+1})-c(k^j_i,T_i)}{\Delta t}+O(\Delta t)
\end{equation}
2. PDE in discretized form.
\begin{equation}
\begin{split}
    \frac{c(k^j_{i+1},T_{i+1})-c(k^j_i,T_i)}{\Delta t}=\frac{\sigma^2}{2}\left(\frac{c(k^{j+1}_i,T_i)-2c(k^{j}_i,T_i)+c(k^{j-1}_i,T_i)}{h^2} - \frac{c(k^{j+1}_i,T_i)-c(k^{j-1}_i,T_i)}{2h} \right)
\end{split}
\end{equation}
\begin{equation}
    c(k^j_{i+1},T_{i+1})=\frac{\sigma^2\Delta t}{4h^2}(2-h)c(k^{j+1}_i,T_i)+(1-\frac{\sigma^2\Delta t}{h^2})c(k^j_i,T_i)+\frac{\sigma^2\Delta t}{4h^2}(2+h)c(k^{j-1}_i,T_i)
\end{equation}
3. Matrix form.

\subsection{Implicit Scheme}
1. Discretize on Maturity Axis.
\begin{equation}
    \frac{\partial c(k^j_i,T_i)}{\partial T}=\frac{c(k^j_{i},T_{i})-c(k^j_{i-1},T_{i-1})}{\Delta t}+O(\Delta t)
\end{equation}
2. PDE in discretized form.
\begin{equation}
\begin{split}
    \frac{c(k^j_{i},T_{i})-c(k^j_{i-1},T_{i-1})}{\Delta t}=\frac{\sigma^2}{2}\left(\frac{c(k^{j+1}_i,T_i)-2c(k^{j}_i,T_i)+c(k^{j-1}_i,T_i)}{h^2} - \frac{c(k^{j+1}_i,T_i)-c(k^{j-1}_i,T_i)}{2h} \right)
\end{split}
\end{equation}
\begin{equation}
    c(k^j_{i+1},T_{i+1})=
\end{equation}
3. Matrix form.

\subsection{Crank-Nicholson Scheme}
1. Discretize on Maturity Axis. \par
2. PDE in discretized form. \par
\begin{equation}
\begin{split}
    \frac{\sigma^2 \Delta t}{8h^2}(2+h)c^{j+1}_{i-1} + (1+ \frac{\sigma^2 \Delta t}{2h^2})c^{j+1}_{i} + \frac{\sigma^2 \Delta t}{8h^2}(2-h)c^{j+1}_{i+1} \\
    = \frac{\sigma^2 \Delta t}{8h^2}(2+h)c^{j}_{i-1} + (1- \frac{\sigma^2 \Delta t}{2h^2})c^{j}_{i} + \frac{\sigma^2 \Delta t}{8h^2}(2-h)c^{j}_{i+1}
\end{split}
\end{equation}

where $\sigma = \sigma^{BS}(k^j_i,T_i)$ or $\sigma^{LV}(k^j_i,T_i)$. \par
3. Matrix form: \par
\begin{equation}
    \begin{bmatrix}
        B^{j}_{1} & C^{j}_{1} & 0 & \cdots & 0 & 0 \\
        A^{j}_{2} & B^{j}_{2} & C^{j}_{2} & \cdots & 0 & 0 \\
        \vdots & \vdots & \vdots & \vdots & \vdots & \vdots \\
        0 & 0 & \cdots & A^{j}_{Nk-3} & B^{j}_{Nk-3} & C^{j}_{Nk-3} \\
        0 & 0 & \cdots & 0 & B^{j}_{Nk-2} & C^{j}_{Nk-2}
    \end{bmatrix}
    \begin{bmatrix}
        c^{j+1}_{1} \\ c^{j+1}_{2} \\ \vdots \\ c^{j+1}_{Nk-3} \\ c^{j+1}_{Nk-2}
    \end{bmatrix} = 
    \begin{bmatrix}
        D^{j}_{1} \\ D^{j}_{2} \\ \vdots \\ D^{j}_{Nk-3} \\ D^{j}_{Nk-2}
    \end{bmatrix}
\end{equation}
where
\begin{equation}
    A^j_i = \frac{\sigma^2 \Delta t}{8h^2}(2+h), \quad B^j_i = (1+ \frac{\sigma^2 \Delta t}{2h^2}), \quad C^j_i = \frac{\sigma^2 \Delta t}{8h^2}(2-h),
\end{equation}
\begin{equation}
     D^j_i = 
     \begin{cases}
        S^j_i - \frac{\sigma^2 \Delta t}{8h^2}(2+h)c^{j+1}_{0} & \text{if } i=1, \\
        S^j_i & \text{if } 0<i<Nk-2, \\
        S^j_i - \frac{\sigma^2 \Delta t}{8h^2}(2-h)c^{j+1}_{Nk-1} & \text{if } i=Nk-2, \\
     \end{cases}
\end{equation}
\begin{equation}
S^j_i = \frac{\sigma^2 \Delta t}{8h^2}(2+h)c^{j}_{i-1} + (1- \frac{\sigma^2 \Delta t}{2h^2})c^{j}_{i} + \frac{\sigma^2 \Delta t}{8h^2}(2-h)c^{j}_{i+1}.
\end{equation}

\end{document}
